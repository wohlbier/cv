\documentclass{muratcan_cv}
\setname{John}{Wohlbier}
\setaddress{Washington, DC USA}
\setmobile{(505) 412-0128}
\setmail{jgwohlbier@sei.cmu.edu}
\setposition{Senior Research Scientist} %ignored for now
\setlinkedinaccount{https://www.linkedin.com/in/john-wohlbier-52a6b5a1/}
\setgithubaccount{https://github.com/wohlbier}
\setthemecolor{red}

\begin{document}
\headerview
\vspace{1ex}

% Summary
\addblocktext{Summary}{%
  Dr.~Wohlbier is a Senior Research Scientist in the Emerging
  Technology Center (ETC) at the Carnegie Mellon University (CMU)
  Software Engineering Institute (SEI). Wohlbier started his
  postdoctoral career at
  Los Alamos
  National Laboratory where he spent over a decade working on
  computational physics for the US Department of Energy Advanced
  Simulation and Computing program. Starting in 2016 Wohlbier spent
  several years supporting DoD HPC programs. Wohlbier leads the
  SEI involvement in the DARPA Software
  Defined Hardware and Domain Specific System on Chip programs. His
  current focus is performance engineering and co-design for data
  intensive
  software. His interests include computation on modern and emerging
  hardware, performance engineering, and computational physics.
}

%Education
\section{Education}
\datedexperience{University of Wisconsin}{}
\explanation{Doctor of Philosophy in Electrical and Computer
  Engineering -- 2003}
\explanationdetail{\center{\bf Thesis --} Nonlinear Distortion and
  Suppression in Traveling Wave Tubes: Insights and Methods.
}
\explanationdetail{\center{\bf
    Advisors --}
  \href{https://prof.ece.wisc.edu/john-booske/}{Professor John Booske} and
  \href{https://www.engineering.iastate.edu/directory/profile/dobson}{Professor
    Ian Dobson}}
\vspace{1ex}
\explanation{Master of Science in Electrical and Computer
  Engineering -- 2000}
\explanationdetail{\center{\bf Thesis --} Modeling and Analysis of a
  Traveling Wave Tube Under Multitone Excitation.}
\vspace{1ex}
\explanation{Bachelor of Science in Electrical and Computer
  Engineering  -- 1993}

%    \explanationdetail{\coloredbullet\ %
%     }

% Experience
\section{Experience}
\datedexperience{Carnegie Mellon University -- Software
  Engineering Institute}{2018 -- present}
\explanation{Senior Research Scientist -- Advanced Computing --
  Emerging Technology Center}
\vspace{1ex}
\explanationdetail{Advanced computing and performance engineering for
  data intensive workloads including machine learning, artificial
  intelligence, and graph analytics. Principal Investigator for CMU
  SEI participation in DARPA Software Defined Hardware and Domain
  Specific System on Chip. Perform expert evaluation of DARPA performer
  tools including ontologies, schedulers, compilers, and novel software
  stacks in new compute platforms. Provide benchmarking analysis and
  performance engineering for DoD relevant workloads. Analyze
  applications for Spiral AI/ML project with long term goal of
  autogeneration of performant code for AI/ML workloads. Serve on CMU
  SEI Technology Council. Evaluate software research proposals from
  across CMU SEI. Assist ETC scientists with proposal development and
  curation.}

\vspace{1ex}
\datedexperience{Engility Corporation}{2016 -- 2018}
\explanation{Computational Scientist}
\vspace{1ex}
\explanationdetail{Performance engineering for US Navy computational
  fluid dynamics codes. Application of performance analysis tools
  including Allinea MAP, Open$|$SpeedShop, Scalasca, Score-p, TAU, and
  Ravel to diagnose performance issues and implement solutions to
  improve performance. Developed best practices guide for application
  of performance engineering methodologies to scientific codes. Worked
  on performance engineering of computational fluid dynamics miniapps
  on Intel Knight's Landing processors. Performed early evaluation of
  performance of new AMD EPYC and Arm 64 architectures using inertial
  confinement fusion code from Rochester Laboratory of Laser
  Energetics. Studied assembly code to correlate results of STREAM and
  HPCG benchmarks with chip microarchitectures.}

\vspace{1ex}
\datedexperience{Los Alamos National Laboratory}{2005 -- 2016}
\explanation{Computational Scientist}
\vspace{1ex}
\explanationdetail{{\bf Multi-physics software development for high
    performance computing.} Active on several multi-physics code
  projects for high performance computing for 10 years. Physics
  include but are not limited to compressible hydrodynamics,
  radiation-matter coupling, and magnetohydrodynamics. Activities
  included coding and optimization for Intel Xeon Phi coprocessors and
  NVidia GPUs, and implementation of distributed memory parallel
  domain decomposition and parallel I/O for an unstructured
  tetrahedral mesh code. C++/C/Fortran. MPI/OpenMP/OpenCL/CUDA.}
\vspace{1ex}
\explanationdetail{
  {\bf Project lead for ``Multi-physics on multi-core''.} Lead
  successful effort to develop multi-physics code for the Cell
  processor in the Roadrunner supercomputer era. Roadrunner, the first
   supercomputer to attain $10^{15}$ FLOPS, was a
  novel heterogeneous architecture and required
  simultaneous programming of three dissimilar processors.
  The compute engine on the Cell processor was a SIMD short
  vector processor, and had significant programmability challenges
  such as the requirement of manual issue of direct memory accesses to
  load registers.}
\vspace{1ex}
\explanationdetail{
  {\bf Data Science Consultant.} Consulted on several data science
  projects with data science company. Developed {\em R} based
  prediction models using large financial and demographic data
  sets. Name of data science company available upon request.}
\vspace{1ex}
\explanationdetail{
  {\bf Postdoctoral and graduate student mentor.} Mentored two
  postdoctoral researchers, both of which are now productive and
  well-respected staff scientists. Mentored and worked with several
  graduate students pursuing degrees in computational physics.
}

\vspace{1ex}
\datedexperience{Los Alamos National Laboratory}{2003 -- 2005}
\explanation{Agnew National Security Postdoctoral Fellow, High
  Power Electrodynamics Group (ISR-6)}
\vspace{1ex}
\explanationdetail{Applied novel analytic and numerical methods to
  study wave breaking, saturation, efficiency, and linearity in
  Traveling Wave Tubes and related devices.
  Collaborated with mathematicians at the University of Wisconsin and
  Princeton University to apply new Eulerian methods of solving
  nonlinear partial
  differential equations to microwave vacuum electronics devices. Solved
  fifty-year-old problem in vacuum electronics and plasma physics.
  Formulated nonlinear space charge wave theory of distortion in
  klystrons.
  Discovered and described fundamental physical mechanism of harmonic
  injection in Traveling Wave Tube Amplifiers.
  Discovered and described fundamental physical mechanism of phase
  distortion in Traveling Wave Tube Amplifiers.
  Derived and tested formulas for growth rates of nonlinear distortion
  products in Traveling Wave Tube Amplifiers.
  Developed Traveling Wave Tube code for research and design of
  Traveling Wave Tubes. Still receive occasional requests for code.
}

\vspace{1ex}
\datedexperience{Department of Engineering Physics -- University of
  Wisconsin}{2003}
\explanation{Postdoctoral Researcher}
\vspace{1ex}
\explanationdetail{Performed theoretical and numerical research in
  stabilization of ballooning instabilities in 3-d plasmas for
  magnetic fusion applications.}

\vspace{1ex}
\datedexperience{Department of Electrical and Computer Engineering --
  University of Wisconsin}{1997 -- 2003}
\explanation{Research Assistant}
\vspace{1ex}
\explanationdetail{Performed research in nonlinear behavior of
  Traveling Wave Tube amplifiers. Research included computer
  simulations and theoretical analyses of Traveling Wave Tubes and
  related devices and resulted in eight refereed journal articles.}

\vspace{1ex}
\datedexperience{Soft Switching Technologies, Middleton,
  Wisconsin}{1994 -- 1997}
\explanation{Engineer}
\vspace{1ex}
\explanationdetail{Electrical and mechanical engineering design for
  start-up company. Involved in all facets of design and production of
  power quality products.
}

\vspace{1ex}
\datedexperience{Department of Electrical and Computer Engineering --
  University of Wisconsin}{1993 -- 1994}
\explanation{Research Assistant}
\vspace{1ex}
\explanationdetail{Performed and published research in semiconductor lasers.}

% Programs
\section{Program Development}
\datedexperience{DARPA ERI}{June, 2020}
\explanation{Principal Investigator}
\vspace{1ex}
\explanationdetail{Proposed performer system evaluation and
  benchmarking work to DARPA in support of
  \href{https://www.darpa.mil/program/software-defined-hardware}{Software
    Defined Hardware},
  \href{https://www.darpa.mil/program/domain-specific-system-on-chip}{Domain
    Specific System on Chip}, and
  \href{https://www.darpa.mil/news-events/2020-03-02}{Data Protection
    in Virtual Environments} programs. Award: \$3.3M base, additional
  \$2.5M optional.}

% Outreach
\section{Professional Service}
\datedexperience{MLSys Workshop: \href{https://resources.sei.cmu.edu/news-events/events/MLSyS-2020-workshop/index.cfm}{Software-Hardware Co-design for
  Machine Learning Workloads}}{March 4, 2020}
\explanation{Co-organizer}
\vspace{1ex}
\explanationdetail{Workshop included talks from DARPA, startup
  companies focused on AI hardware, universities, national
  laboratories, and established chip vendors. The keynote presentation
  titled ``HW-SW Co-design: Dogs and cats living together, mass
  hysteria!'' was given by DARPA program manager Dr. Tom
  Rondeau.
  %Dr. Rondeau detailed codesign investments DARPA is making
  %in the Software Defined Hardware and Domain Specific System on a
  %Chip programs. Rondeau said ``Hardware-Software codesign is a
  %critical concept in designing systems for machine learning to enable
  %better performance at lower power. DARPA has recognized this
  %important concept and has launched a number of programs within its
  %Electronics Resurgence Initiative that address these issues with
  %both defense and commercial industries. The timely HW-SW Codesign
  %workshop allowed us to communicate with an impressive list of
  %speakers and attendees that will help us further expand our
  %engagement with this community.''
}
\vspace{1ex}
\datedexperience{SC18 Birds of a Feather:
  \href{https://sc18.supercomputing.org/proceedings/bof/bof_pages/bof178.html}{The
    ARM HPC Experience: From Testbeds to Exascale}}{November 14, 2018}
\explanation{Co-organizer}
\vspace{1ex}
\explanationdetail{Lightning talks and a panel on the emerging 64-bit
  Arm architecture for the datacenter and supercomputers. Presenters
  included leading HPC organizations such as RIKEN, Sandia National
  Laboratories, US Naval Research Laboratory,
  and the University of Bristol.
}

% Skills
\section{Skills}
%
\newcommand{\skillone}{\createskill{Programming Languages}{
    Assembly \cpshalf
    C \cpshalf
    C++  \cpshalf
    Fortran \cpshalf
    LLVM IR \cpshalf
    Python}}
%
\newcommand{\skilltwo}{\createskill{Software Development}{
    Docker \cpshalf
    Emacs \cpshalf
    git \cpshalf
    LLVM \cpshalf
    Spack \cpshalf
    Singularity}}
%
\newcommand{\skillthree}{\createskill{Frameworks \& Libraries}{
    CUDA \cpshalf
    MPI \cpshalf
    Numpy \cpshalf
    OpenCL \cpshalf
    OpenMP \cpshalf
    PyTorch \cpshalf
    Tensorflow}}
%
\newcommand{\skillfour}{\createskill{Performance Engineering}{
    Intel VTune \& Advisor \cpshalf
    Arm Forge \cpshalf
    TAU Commander}}
%
\createskills{\skillone, \skilltwo, \skillthree, \skillfour}

\section{Publications}
\begin{itemize}
\footnotesize
\item Walden, A., Nielsen, E.J., Zubair, M., Linford, J.C., Wohlbier,
  J.G., Luitjens, J.P., Orender, J., Beekman, I., Khuvis, S. and Shende,
  S.S., 2017, November. Unstructured-Grid CFD Algorithms on Many-Core
  Architectures. In {\it Technical Research Posters of International
    Supercomputing Conference for High Performance Computing, Networking,
    Storage, and Analysis.} SC (Vol. 17).
\item Nystrom, W.D., Bergen, B., Bird, R.F., Bowers, K.J., Daughton,
  W.S., Guo, F., Li, H., Nam, H.A., Pang, X., Rust III, W.N. and
  Wohlbier, J., 2016. Performance of VPIC on Trinity. APS, 2016,
  pp.JP10-097.
\item Charest, M.R., Canfield, T.R., Morgan, N.R., Waltz, J. and
  Wohlbier, J.G., 2015. A high-order vertex-based central ENO
  finite-volume scheme for three-dimensional compressible
  flows. {\it Computers \& Fluids}, 114, pp.172-192.
\item Morgan, N.R., Waltz, J.I., Burton, D.E., Charest, M.R.,
  Canfield, T.R. and Wohlbier, J.G., 2015. A point-centered arbitrary
  Lagrangian Eulerian hydrodynamic approach for tetrahedral
  meshes. {\it Journal of Computational Physics}, 290, pp.239-273.
\item Waltz, J., Wohlbier, J.G., Risinger, L.D., Canfield, T.R.,
  Charest, M.R.J., Long, A.R. and Morgan, N.R., 2015. Performance
  analysis of a 3D unstructured mesh hydrodynamics code on multi‐core
  and many‐core architectures. {\it International Journal for Numerical
    Methods in Fluids}, 77(6), pp.319-333.
\item Morgan, N.R., Waltz, J.I., Burton, D.E., Charest, M.R.,
  Canfield, T.R. and Wohlbier, J.G., 2015. A Godunov-like
  point-centered essentially Lagrangian hydrodynamic approach. {\it Journal
    of Computational Physics}, 281, pp.614-652.
\item Waltz, J., Morgan, N.R., Canfield, T.R., Charest, M.R.J. and
  Wohlbier, J.G., 2014. A nodal Godunov method for Lagrangian shock
  hydrodynamics on unstructured tetrahedral grids. {\it International
    Journal for Numerical Methods in Fluids}, 76(3), pp.129-146.
\item Waltz, J., Canfield, T.R., Morgan, N.R., Risinger, L.D. and
  Wohlbier, J.G., 2014. Manufactured solutions for the
  three-dimensional Euler equations with relevance to Inertial
  Confinement Fusion. {\it Journal of Computational Physics}, 267,
  pp.196-209.
\item Waltz, J., Morgan, N.R., Canfield, T.R., Charest, M.R.,
  Risinger, L.D. and Wohlbier, J.G., 2014. A three-dimensional
  finite element arbitrary Lagrangian–Eulerian method for shock
  hydrodynamics on unstructured grids. {\it Computers \& Fluids}, 92,
  pp.172-187.
\item Waltz, J., Canfield, T.R., Morgan, N.R., Risinger, L.D. and
  Wohlbier, J.G., 2013. Verification of a three-dimensional
  unstructured finite element method using analytic and manufactured
  solutions. {\it Computers \& Fluids}, 81, pp.57-67.
\item Fatenejad, M., Fryxell, B., Wohlbier, J., Myra, E., Lamb, D.,
  Fryer, C. and Graziani, C., 2013. Collaborative comparison of
  simulation codes for high-energy-density physics applications. {\it High
    Energy Density Physics}, 9(1), pp.63-66.
\item Masser, T.O., Wohlbier, J.G. and Lowrie, R.B., 2011. Shock wave
  structure for a fully ionized plasma. {\it Shock waves}, 21(4),
  pp.367-381.
\item McClarren, R.G. and Wohlbier, J.G. Solutions for
  ion-electron-radiation coupling with radiation and electron
  diffusion. {\it J.~Quant.~Spectrosc.~Radiat.~Transfer}, 112, pp.119-130
  (2011).
\item Wohlbier, J.G., 2005. Phase distortion mechanisms in linear beam
  vacuum devices. {\it IEEE transactions on plasma science}, 33(3),
  pp.1031-1035.
\item Wohlbier, J.G., Jin, S. and Sengele, S., 2005. Eulerian
  calculations of wave breaking and multivalued solutions in a
  traveling wave tube. {\it Physics of Plasmas}, 12(2), p.023106.
\item Wohlbier, J.G. and Booske, J.H., 2005. Nonlinear space charge
  wave theory of distortion in a klystron. {\it IEEE Transactions on
    Electron Devices}, 52(5), pp.734-741.
\item Singh, A., Scharer, J.E., Booske, J.H. and Wohlbier, J.G.,
  2005. Second-and third-order signal predistortion for nonlinear
  distortion suppression in a TWT. {\it IEEE Transactions on Electron
    Devices}, 52(5), pp.709-717.
\item Section 3, ``Theoretical Principles,'' in Chapter 9,
  ``How to Achieve Linear Amplification,'' in Barker, R.J., Luhmann,
  N.C., Booske, J.H. and Nusinovich, G.S., 2005. Modern microwave and
  millimeter-wave power electronics (p. 872).
\item Singh, A., Wohlbier, J.G., Booske, J.H. and Scharer, J.E.,
  2004. Experimental verification of the mechanisms for nonlinear
  harmonic growth and suppression by harmonic injection in a traveling
  wave tube. {\it Physical Review Letters}, 92(20), p.205005.
\item Li, X., Wöhlbier, J.G., Jin, S. and Booske, J.H., 2004. Eulerian
  method for computing multivalued solutions of the Euler-Poisson
  equations and applications to wave breaking in klystrons. {\it Physical
    Review E}, 70(1), p.016502.
\item Wohlbier, J.G. and Booske, J.H., 2004. Mechanisms for phase
  distortion in a traveling wave tube. {\it Physical Review E}, 69(6),
  p.066502.
\item Wohlbier, J.G., Booske, J.H. and Dobson, I., 2004. On the
  physics of harmonic injection in a traveling wave tube. {\it IEEE
    Transactions on Plasma Science}, 32(3), pp.1073-1085.
\item Wohlbier, J.G., Dobson, I. and Booske, J.H., 2002. Generation
  and growth rates of nonlinear distortions in a traveling wave
  tube. {\it Physical Review E}, 66(5), p.056504.
\item Wohlbier, J.G., Booske, J.H. and Dobson, I., 2002. The
  multifrequency spectral Eulerian (MUSE) model of a traveling wave
  tube. {\it IEEE transactions on plasma science}, 30(3), pp.1063-1075.
\item Zhang, T., Wohlbier, J.G., Choquette, K.D. and Tabatabaie, N.,
  1995. Microcavity vacuum-field configuration and the spontaneous
  emission power. {\it IEEE Journal of Selected Topics in Quantum
    Electronics}, 1(2), pp.606-615.
\end{itemize}

\section{Patents}
\begin{itemize}
  \footnotesize
\item U.S. Patent 6,087,916. Cooling of coaxial winding transformers in
  high power applications. N.H. Kutkut, D.M. Divan, J.G. Wohlbier,
  R.W. Gascoigne.
\end{itemize}

%{\bf\large Teaching Experience}
%\begin{itemize}
%\item Guest Lecturer, Electrical and Computer Engineering,
%  University of Wisconsin--Madison, 1998--2003.\\
%  Frequently lecture undergraduate electromagnetic theory courses.
%\item Teaching Assistant, Physics, University of Wisconsin--Madison,
%  1999.\\
%  Prepared and led discussion and laboratory sessions for
%  undergraduate Physics class in electromagnetism. Received highest
%  evaluation of all Teaching Assistants (9) in course.
%\item Teaching Assistant, Electrical and Computer Engineering,
%  University of Wisconsin--Madison, 1993--1994.\\
%  Developed and assisted advanced undergraduate
%  laboratories. TA of the year award.
%\end{itemize}

\createfootnote
\end{document}
