\documentclass[12pt]{article}
\setlength{\oddsidemargin}{-0.2in}
\setlength{\textwidth}{6.5in}
%\setlength{\topmargin}{-0.5in}
%\setlength{\textheight}{9.5in}

\def\rage{{\sf RAGE}}
\def\ocl{{\sf OpenCL}}

%\usepackage{fancyhdr}
%\renewcommand{\headrulewidth}{0.0mm}
%\renewcommand{\footrulewidth}{0.0mm}
%\fancyhf{}
%\fancyhead[R]{January 2006}
%\pagestyle{fancy}
%\renewcommand{\labelitemii}{$\Rightarrow$}

\begin{document}
%\begin{center}
  {\noindent\bf John G. Wohlbier\\
    High Performance Computing Consultant\\
    Ci Software Associates}\\
%  PO Box 1663, MS D413\\
%  Los Alamos, NM 87545\\
%  (505)664-0794\\
%  {\tt wohlbier@lanl.gov}\\
%  (505)664-0794 \\
%  {\tt wohlbier@lanl.gov}\\
  (505)412-0128 \\
  {\tt johnwohlbier@gmail.com}\\
%\end{center}

{\noindent\bf Education and Training}
\begin{itemize}
\item Ph.D. in Electrical and Computer Engineering, University
  of Wisconsin--Madison, May 2003.
\item Master of Science in Electrical Engineering,
  University of Wisconsin--Madison, May 2000.
\item Bachelor of Science in Electrical Engineering,
  University of Wisconsin--Madison, August 1993.\\
\end{itemize}

{\noindent\bf Appointments}

\noindent{\bf 2016--present:} High Performance Computing Consultant,
Ci Software Associates.

\noindent{\bf 2005--2016:} Technical Staff Member, Computational Physics
and Methods (CCS-2), Los Alamos National Laboratory.

\noindent{\bf 2003--2005:} Agnew National Security Post-doctoral Fellow, High
Power Electrodynamics Group (ISR-6), Los Alamos National Laboratory.

\noindent{\bf 2003--2003:} Postdoctoral Researcher, Department of Engineering
Physics, University of Wisconsin, Madison.\\

{\noindent\bf Experience}

\begin{itemize}
\item {\bf High Performance Computing Consultant.} Performance engineering
for US Navy computational fluid dynamics codes. Application of performance
analysis tools including Allinea MAP, Open$|$SpeedShop, Scalasca, Score-p, TAU,
and Ravel to diagnose performance issues and implement solutions to improve
performance. Developed best practices guide for application of performance
engineering methodologies to scientific codes. Worked on performance
engineering of computational fluid dynamics miniapps on Intel Knight's
Landing processors.
\item {\bf Multi-physics software development for high performance
  computing.} Active on several multi-physics code projects for high
performance computing for 10 years. Physics include but are not
limited to compressible hydrodynamics, radiation-matter coupling, and
magnetohydrodynamics. Recent
activities include coding and optimization for Intel Xeon Phi
coprocessors and NVidia GPUs, and implementation of distributed memory
parallel domain decomposition and parallel
I/O for an unstructured tetrahedral mesh
code. C++/C/Fortran. MPI/OpenMP/OpenCL/CUDA. 
\item {\bf Project lead for ``Multi-physics on multi-core''.} Lead
  successful effort to develop multi-physics code for the Cell
  processor in the Roadrunner supercomputer era. Roadrunner was a
  radically different heterogeneous architecture that required
  simultaneous use of the three very different types of processors on
  the system. The compute engine on the Cell processor was a short
  vector processor similar to what is emerging in current Intel
  hardware, but had additional barriers to use such as the need to
  specifically issue direct memory accesses (DMA's) to load
  registers.
\item {\bf Data Science Consultant.} Consulted on several data science
  projects with data science company. Developed {\em R} based
  prediction models using large financial and demographic data
  sets. Name of data science company available upon request.
\item {\bf Post-doc and graduate student mentor.} Mentored two
  post-doctoral researchers, both of which are now productive and well
  respected staff scientists. Mentored and worked with several
  graduate students pursuing degrees in computational physics.
\end{itemize}

\noindent{\bf Publications}
\begin{itemize}
\item N.R.~Morgan, J.~Waltz, D.E.~Burton, M.R.J.~Charest,
  T.R.~Canfield, and J.G.~Wohlbier, A point-centered arbitrary
  Lagrangian Eulerian hydrodynamic approach for tetrahedral meshes.
  {\it Journal of Computational Physics}, 290:239-273 (2015)
\item J.~Waltz, J.G.~Wohlbier, L.D.~Risinger, T.R.~Canfield,
  M.R.J.~Charest, A.R.~Long, and N.R.~Morgan, Performance analysis of
  a 3D unstructured mesh hydrocode on multi- and many-core
  architectures. {\it International Journal for Numerical Methods in
    Fluids}, 77:319-333 (2015).
\item M.R.J.~Charest, T.R.~Canfield, N.R.~Morgan,
  L.D.~Risinger,~J.~Waltz, and J.G.~Wohlbier. A high-order
  vertex-based central ENO finite volume scheme for three-dimensional
  compressible flows. submitted to {\it Computers \& Fluids}, (2014).
\item N.R.~Morgan, J.~Waltz, D.E.~Burton, M.R.J.~Charest,
  T.R.~Canfield, and J.G.~Wohlbier. A Godunov-like point-centered
  essentially Lagrangian hydrodynamic approach. {\it Journal of
    Computational Physics}, 281:614-652 (2014).
\item J.~Waltz, N.R.~Morgan, T.R.~Canfield, M.R.J.~Charest, and
  J.G.~Wohlbier, A nodal Godunov method for Lagrangian shock
  hydrodynamics on unstructured tetrahedral grids.
  {\it International Journal for Numerical Methods in Fluids},
  76:129-146 (2014).
\item J.~Waltz, T.R.~Canfield, N.R.~Morgan, L.D.~Risinger, and
  J.G.~Wohlbier. Manufactured solutions for the three-dimensional
  Euler equations with relevance to Inertial Confinement Fusion.
  {\it Journal of Computational Physics}, 267(15):196-209 (2014).
\item J.~Waltz, N.R.~Morgan, T.R.~Canfield, M.R.J.~Charest,
  L.D.~Risinger, and J.G.~Wohlbier. A three-dimensional finite
  element arbitrary Lagrangian-Eulerian method for shock hydrodynamics
  on unstructured grids. {\it Computers \& Fluids}, 92(20):172-187
  (2014).
\item J.~Waltz, T.R.~Canfield, N.R.~Morgan, L.D.~Risinger, and
  J.G.~Wohlbier. Verification of a three-dimensional unstructured
  finite element method using analytic and manufactured solutions.
  {\it Computers \& Fluids}, 81(20):57-67 (2013).
\item M.~Fatenejad, B.~Fryxell, J.~Wohlbier, E.~Myra, D.~Lamb,
  C.~Fryer, C.~Graziani. Collaborative comparison of simulation codes
  for high-energy-density physics applications. {\it High Energy
    Density Physics}, 9(1):63-66 (2013).
\item T.O.~Masser, J.G.~Wohlbier, and R.B.~Lowrie. Shock wave structure
  for a fully ionized plasma. {\it Shock Waves}, 21:367--381 (2011).
\item R.G.~McClarren and J.G.~Wohlbier. Solutions for
  ion-electron-radiation coupling with radiation and electron
  diffusion. {\it J.~Quant.~Spectrosc.~Radiat.~Transfer}, 112:119--130
  (2011).
\item J.G.~Wohlbier. Phase distortion mechanisms in linear beam
  vacuum devices. {\it IEEE Trans.~Plasma Sci.}, Vol.~33, no.~3,
  2005.
\item J.G.~Wohlbier, S.~Jin, S.~Sengele. Eulerian calculations of
  wave breaking and multi-valued solutions in a traveling wave
  tube. {\it Physics of Plasmas} {\bf 12}, 023106 (2005).
\item J.G.~Wohlbier and J.H.~Booske. Nonlinear space charge wave theory
  of distortion in a klystron. {\it IEEE Trans.~Electron Devices,}
  Vol.~52, no.~5, 2005.
\item A.~Singh, J.E.~Scharer, J.H.~Booske, and J.G.~Wohlbier.
  Second and third-order signal injection for nonlinear
  distortion suppression in a traveling wave tube. {\it IEEE
    Trans.~Electron Devices,} Vol.~52, no.~5, 2005.
\item A.~Singh, J.G.~Wohlbier, J.H.~Booske, and
  J.E.~Scharer. Experimental Verification of the Mechanisms for
  Nonlinear Harmonic Growth and Suppression by Harmonic Injection in a
  Traveling Wave Tube. {\it Phys.~Rev.~Lett.} 92, 205005 (2004).
\item X.~Li, J.G.~Wohlbier, S.~Jin, and J.H.~Booske.
  Eulerian Method for Computing Multi-valued solutions of the Euler-Poisson
  Equations and Application to Wave Breaking in Klystrons.
  {\it Phys.~Rev.~E} 70, 016502 (2004).
\item J.G.~Wohlbier and J.H.~Booske.
  Mechanisms of Phase Distortion in a Traveling Wave Tube.
  {\it Phys.~Rev.~E} 69, 066502 (2004).
\item J.G.~Wohlbier, J.H.~Booske, and I.~Dobson.
  On the Physics of Harmonic Injection in a Traveling Wave Tube.
  {\it IEEE Trans.~Plasma Sci.}, Vol.~32, No.~3, (2004).
\item J.G.~Wohlbier, I.~Dobson, and J.H.~Booske.
  Generation and growth rates of nonlinear distortions in a traveling
  wave tube. {\it Phys.~Rev.~E} 66, 56504 (2002).
\item J.G.~Wohlbier, J.H.~Booske, and I.~Dobson.
  The Multifrequency Spectral Eulerian (MUSE) Model of a Traveling Wave
  Tube. {\it IEEE Trans.~Plasma Sci.} Vol.~30, no.~3, June 2002.
\item T.~Zhang, J.G.~Wohlbier, K.D.~Choquette, N.~Tabatabaie.
  Microcavity Vacuum-Field Configuration and the Spontaneous Emission
  Power. {\it IEEE Journal on Selected Topics in Quantum Electronics.}
  Vol.~1, no.~2, pp.~601--605, 1995.
\end{itemize}

%\noindent {\bf Synergystic Professional Activities}
%
%\begin{itemize}
%\item Leading {\ocl}~developement activity within \rage
%\item \rage\ code developer, Los Alamos National Laboratory. Develop
%multi-physics algorithms in the \rage\ code, including the initial
%implementation of a 3T model in \rage.\\
%\end{itemize}
%
%\noindent {\bf Collaborators and Other Affiliations}
%
%\noindent FLASH center (U.~Chicago), The \rage\ code team (LANL), R.B.~Lowrie
%(LANL), R.G.~McClarren (Texas A\&M), B.~Bergen (LANL), T.O.~Masser
%(LANL), M.~Calef (LANL).



%{\bf \large Research Experience}
%\begin{itemize}
%\item Technical Staff Member, Los Alamos National Laboratory,
%  Continuum Dynamics Group (CCS-2). 2005 -- present.\\
%  Develop code and numerical methods targeted for, but not limited to,
%  the \rage\ code. \rage\ is the LANL
%  flagship AMR Eulerian code for the Department of Energy's Advanced
%  Simulation and Computing (ASC) program:
%  \begin{itemize}
%  \item Developed the \xrage\ library, a set of \rage\ extensions for
%    linking advanced and alternate physics modules. \xrage\ highlights
%    include use of sophisticated parallel communication primitives to
%    enable large AMR stencils, and \rage\ interface to enable use of any
%    cell centered or vertex based hydro scheme. The \xrage\ interface
%    to \rage\ has been used to link to Paul Woodward's PPM-PPB
%    multi-fluid developmental hydro, a vertex based hydro algorithm
%    developed in LANL's X division, and \GPTe\ (below).
%  \item Developed \RL, a stand alone library of Riemann solvers for
%    hydrodyanmics applications. \RL\ is intended to grow to include a
%    large library of (possibly contributed) Riemann solvers.
%  \item Developed \GPTe, a stand alone library of pressure temperature
%    equilibrium Godunov hydrodyanmics solvers. \GPTe\ uses, but is not
%    limited to, \RL, and is
%    intended to grow to include a
%    large library of (possibly contributed) pressure temperature
%    equilibrium hydro solvers. \GPTe\ has been linked to \rage\
%    through the \xrage\ interface.
%  \item Performing research on multi-temperature hydro models for
%    implementation and eventual use in the \rage\ code.
%  \end{itemize}
%\end{itemize}

%{\bf\large Postdoctoral and Graduate Research Experience}
%\begin{itemize}
%\item Agnew National Security Postdoctoral Fellow, Los Alamos National
%  Laboratory, 2003--2005.
%  Applied novel analytic and numerical methods to study wave
%  breaking, saturation,
%  efficiency, and linearity in Traveling Wave Tubes and related
%  devices. Work was on forefront of methods in nonlinear
%  waves and shock physics.
%  %Large scale computations developed and
%  %performed on Los Alamos parallel computing resources.
%\item Postdoctoral Scholar, Nuclear Engineering and Engineering
%  Physics, University of Wisconsin--Madison, 2003.
%  Performed theoretical and numerical research in stabilization
%  of ballooning instabilities in 3-d plasmas for magnetic fusion
%  applications. Work towards a publication is ongoing.
%\item Research Assistant, Electrical and Computer Engineering,
%  University of Wisconsin--Madison, 1997--2003.
%  Performed research in nonlinear behavior of Traveling Wave Tube
%  amplifiers. Research included computer simulations and theoretical
%  analyses of Traveling Wave Tubes and related devices and resulted in
%  eight refereed journal articles.
%\item Research Assistant, Electrical and Computer Engineering,
%  University of Wisconsin--Madison, 1993--1994.
%  Performed and published research in semiconductor lasers.
%\end{itemize}

%{\bf\large Education}
%\begin{itemize}
%\item Ph.D. in Electrical and Computer Engineering, University
%  of Wisconsin--Madison, May 2003.
%  Thesis: Nonlinear Distortion and Suppression in Traveling Wave
%  Tubes: Insights and Methods. Advisors: John Booske and Ian Dobson.
%\item Master of Science in Electrical Engineering,
%  University of Wisconsin--Madison, May 2000.
%  Thesis: Modeling and Analysis of a Traveling Wave Tube Under Multitone
%  Excitation.
%\item Bachelor of Science in Electrical Engineering,
%  University of Wisconsin--Madison, August 1993.
%\end{itemize}

%{\bf\large Teaching Experience}
\begin{itemize}
\item Guest Lecturer, Electrical and Computer Engineering,
  University of Wisconsin--Madison, 1998--2003.\\
  Frequently lecture undergraduate electromagnetic theory courses.
\item Teaching Assistant, Physics, University of Wisconsin--Madison,
  1999.\\
  Prepared and led discussion and laboratory sessions for
  undergraduate Physics class in electromagnetism. Received highest
  evaluation of all Teaching Assistants (9) in course.
\item Teaching Assistant, Electrical and Computer Engineering,
  University of Wisconsin--Madison, 1993--1994.\\
  Developed and assisted advanced undergraduate
  laboratories. TA of the year award.
\end{itemize}


%{\bf\large Industrial Experience}
%\begin{itemize}
%\item Engineer, Soft Switching Technologies, Middleton,
%  Wisconsin, 1994--1997.\\
%  Electrical and mechanical engineering design for start-up company.
%  Involved in all facets of design and production of power quality
%  products.
%\end{itemize}
%
%%\clearpage
%%\pagebreak
%\fancyhead[R]{John G.~W\"ohlbier}


%{\bf\large Principal Research Accomplishments}
%\begin{itemize}
%\item Collaborated with mathematicians (University of Wisconsin and Princeton
%  University) to apply new Eulerian methods of solving nonlinear partial
%  differential equations to microwave vacuum electronics devices. Solves
%  fifty year old problem in vacuum electronics and plasma physics.
%\item Formulated nonlinear space charge wave theory of distortion in
%  klystrons.
%\item Discovered and described fundamental physical mechanism of harmonic
%  injection in Traveling Wave Tube Amplifiers.
%\item Discovered and described fundamental physical mechanism of phase
%  distortion in Traveling Wave Tube Amplifiers.
%\item Derived and tested formulas for growth rates of nonlinear distortion
%  products in Traveling Wave Tube Amplifiers.
%\item Developed Traveling Wave Tube code for research and design of
%  Traveling Wave Tubes. Code internationally known and used. See
%  {\tt http://www.lmsuite.org}
%\end{itemize}

%{\bf\large Research Interests}
%\begin{itemize}
%\item Computational physics and large scale computing
%\item Plasma physics
%\item Microwave electronics
%\item Dynamical systems
%\item Applied mathematics
%\end{itemize}

%{\bf\large Submitted Manuscripts}
%\begin{enumerate}
%\end{enumerate}

%{\bf\large Refereed Journal Publications}
%\begin{enumerate}
%\item J.G. W\"ohlbier. Phase distortion mechanisms in linear beam
%  vacuum devices. {\it IEEE Trans.~Plasma Sci.}, Vol.~33, no.~3,
%  2005.
%  \\$\Rightarrow$0 citations.
%\item J.G.~W\"ohlbier, S.~Jin, S.~Sengele. Eulerian calculations of
%  wave breaking and multi-valued solutions in a traveling wave
%  tube. {\it Physics of Plasmas} {\bf 12}, 023106 (2005).
%  \\$\Rightarrow$1 citation.
%\item J.G. W\"ohlbier and J.H. Booske. Nonlinear space charge wave theory
%  of distortion in a klystron. {\it IEEE Trans.~Electron Devices,}
%  Vol.~52, no.~5, 2005.
%  \\$\Rightarrow$1 citation.
%\item A. Singh, J.E. Scharer, J.H. Booske, and J.G. W\"ohlbier.
%  Second and third-order signal injection for nonlinear
%  distortion suppression in a traveling wave tube. {\it IEEE
%    Trans.~Electron Devices,} Vol.~52, no.~5, 2005.
%  \\$\Rightarrow$0 citations.
%\item A. Singh, J.G. W\"ohlbier, J.H. Booske, and
%  J.E. Scharer. Experimental Verification of the Mechanisms for
%  Nonlinear Harmonic Growth and Suppression by Harmonic Injection in a
%  Traveling Wave Tube. {\it Phys.~Rev.~Lett.} 92, 205005 (2004).
%  \\$\Rightarrow$2 citations.
%\item X. Li, J.G. W\"ohlbier, S. Jin, and J.H. Booske.
%  Eulerian Method for Computing Multi-valued solutions of the Euler-Poisson
%  Equations and Application to Wave Breaking in Klystrons.
%  {\it Phys.~Rev.~E} 70, 016502 (2004).
%  \\$\Rightarrow$3 citations.
%\item J.G. W\"ohlbier and J.H. Booske.
%  Mechanisms of Phase Distortion in a Traveling Wave Tube.
%  {\it Phys.~Rev.~E} 69, 066502 (2004).
%  \\$\Rightarrow$3 citations.
%\item J.G. W\"ohlbier, J.H. Booske, and I. Dobson.
%  On the Physics of Harmonic Injection in a Traveling Wave Tube.
%  {\it IEEE Trans.~Plasma Sci.}, Vol.~32, No.~3, (2004).
%  \\$\Rightarrow$3 citations.
%\item J.G. W\"ohlbier, I. Dobson, and J.H. Booske.
%  Generation and growth rates of nonlinear distortions in a traveling
%  wave tube. {\it Phys.~Rev.~E} 66, 56504 (2002).
%  \\$\Rightarrow$6 citations.
%\item J.G. W\"ohlbier, J.H. Booske, and I. Dobson.
%  The Multifrequency Spectral Eulerian (MUSE) Model of a Traveling Wave
%  Tube. {\it IEEE Trans.~Plasma Sci.} Vol.~30, no.~3, June 2002.
%  \\$\Rightarrow$11 citations.
%\item T. Zhang, J.G. W\"ohlbier, K. D. Choquette, N. Tabatabaie.
%  Microcavity Vacuum-Field Configuration and the Spontaneous Emission
%  Power. {\it IEEE Journal on Selected Topics in Quantum Electronics.}
%  Vol.~1, no.~2, pp.~601--605, 1995.
%  \\$\Rightarrow$8 citations.
%\end{enumerate}
%\clearpage



%{\bf\large Manuscripts in Preparation}

%{\bf\large Book Chapter}
%\begin{itemize}
%\item Section 3, ``Theoretical Principles,'' in Chapter 9,
%  ``How to Achieve Linear Amplification,'' in Barker, R.J., Booske,
%  J.H., Luhmannn, N.C., Nusinovich, G.S. (2005).
%  {\it Modern Microwave and Millimeter Wave Power Electronics:} IEEE
%  Press.
%\end{itemize}
%
%{\bf\large Software}
%\begin{itemize}
%\item J.G.~W\"ohlbier. LATTE/MUSE Numerical Suite ({\em lmsuite}).
%  A suite of 1-d nonlinear Traveling Wave Tube codes provided
%  to the microwave vacuum electronics community for research purposes.
%  Codes are used by several international researchers.
%  Available for download at {\tt http://www.lmsuite.org}
%\end{itemize}
%
%{\bf\large Computing Skills}
%\begin{itemize}
%\item Linux, Windows and Macintosh, parallel computer programming with
%  the Message Passing Interface (MPI), C++, C, Fortran, object oriented
%  programming in most modern languages.
%\end{itemize}
%
%{\bf\large Honors and Awards}
%\begin{itemize}
%\item 2003 Agnew National Security Postdoctoral Fellow, Los Alamos
%  National Laboratory. Highly competitive fellowship, only three
%  awarded out of nearly 400 post-docs.
%\item 1993 Teaching Assistant of the Year Award. University of
%  Wisconsin--Madison.
%  An award decided by student vote in the department of Electrical and
%  Computer Engineering.
%\end{itemize}
%
%{\bf\large Patent}
%\begin{itemize}
%\item U.S. Patent 6,087,916. Cooling of coaxial winding transformers in
%  high power applications. N.H. Kutkut, D.M. Divan, J.G. Wohlbier,
%  R.W. Gascoigne.
%\end{itemize}
%
%%\pagebreak
%%{\bf\large Conference Papers and Presentations}
\begin{itemize}
%\item S.R. Hudson {\it et al.} Influence of pressure-gradient and
%  shear on ballooning stability in stellarators.
%  20$^{\rm th}$ International Atomic Energy Agency Fusion Energy
%  Conference, November 2004.
\item J.G. W\"ohlbier, S. Sengele. New Eulerian methods for nonlinear
  wave equations: computing multi-valued solutions in a traveling wave
  tube. 46$^{\rm th}$ Annual Meeting of the American Physical Society
  Division of Plasma Physics, November 2004.
\item J.G. W\"ohlbier, C.C. Hegna. Finite Larmor radius stabilization
  of ideal ballooning instabilities in 3-d plasmas.
  Innovative Confinement Concepts Workshop, May 2004.
\item J.G. W\"ohlbier, J.H. Booske. A modal description of
  intermodulation injection in a klystron. Fifth IEEE International
  Vacuum Electronics Conference, April 2004.
\item A. Singh, J.E. Scharer, J.G. W\"ohlbier,
  J.H. Booske. Sensitivity of harmonic injection and its spatial
  evolution for nonlinear distortion suppression in a TWT.
  Fifth IEEE International Vacuum Electronics Conference, April 2004.
\item J.G. W\"ohlbier, J.H. Booske, I. Dobson, A. Singh,
  J.E. Scharer. A new look at the nonlinear physics of Traveling Wave
  Tubes. 45$^{\rm th}$ Annual Meeting of the American Physical Society
  Division of Plasma Physics, October 2003.
\item J.G. W\"ohlbier, C.C. Hegna. Finite Larmor radius stabilization
  of ideal ballooning modes in three-dimensional geometry.
  45$^{\rm th}$ Annual Meeting of the American Physical Society
  Division of Plasma Physics, October 2003.
\item C.C. Hegna, J.G. W\"ohlbier. Finite Larmor radius stabilization
  of ideal ballooning modes in three-dimensional geometry.
  14$^{\rm th}$ International Stellarator Workshop, September 2003.
\item S. Bhattacharjee {\it et al.} Folded waveguide traveling wave
  tube sources for THz radiation. 30$^{\rm th}$ International
  Conference on Plasma Science, June 2003.
\item A. Singh, J.G. W\"ohlbier, J.E. Scharer, J.H. Booske. Injection
  schemes for TWT linearization. 30$^{\rm th}$ International
  Conference on Plasma Science, June 2003.
\item J.G. W\"ohlbier, M.C. Converse, J. Plouin, A. Rawal, A. Singh,
 J.H. Booske. LATTE/MUSE numerical suite: an open source teaching and
 research code for traveling wave tube amplifiers. 30$^{\rm th}$
 International Conference on Plasma Science, June 2003.
\item S. Bhattacharjee {\it et al.} Investigations of folded waveguide
  TWT oscillators for THz radiation. Fourth IEEE International Vacuum
  Electronics Conference, May 2003.
\item J.G. W\"ohlbier, J.H. Booske, I. Dobson.
  The Physics of Phase Nonlinearity in a Traveling Wave Tube.
  44$^{\rm th}$ Annual Meeting of the American Physical Society
  Division of Plasma Physics, November 2002.
\item X. Li, J.G. W\"ohlbier, S. Jin, J.H. Booske.
  An Eulerian Method for Computing Multi-valued Solutions of the
  Euler-Poisson Equations.
  44$^{\rm th}$ Annual Meeting of the American Physical Society
  Division of Plasma Physics, November 2002.
\item J.G. W\"ohlbier, J.H. Booske, I. Dobson.
  Comparison of Transfer Curve Induced Distortions to Distortions in
  Nonlinear Physical TWT Models.
  28$^{\rm th}$ IEEE Conference on Plasma Science, June 2002.
\item J.G. W\"ohlbier, J.H. Booske, I. Dobson.
  The Physics of Harmonic Injection in a TWT.
  3$^{\rm rd}$ IEEE International Vacuum Electronics Conference, April
  2002.
\item J.G. W\"ohlbier, J.H. Booske, I. Dobson.
  Finite Bandwidth and Space Charge Effects in the MUSE Model. IEEE
  Pulsed Power Plasma Science Conference, June 2001.
\item J.G. W\"ohlbier, J.H. Booske, I. Dobson.
  New Nonlinear Multifrequency TWT Model.
  2$^{\rm nd}$ IEEE International Vacuum Electronics Conference, April 2001.
\item J.G. W\"ohlbier, I. Dobson, J.H. Booske.
  Four Multitoned Traveling Wave Tube Models.
  27$^{\rm th}$ IEEE Conference on Plasma Science, June 2000.
\item M.A. Wirth {\it et al.} Investigations of non-linear spectral
  behavior in multi-toned helix traveling wave tubes. 27$^{\rm th}$
  IEEE International Conference on Plasma Science, June 2000.
\item J.G. W\"ohlbier, J.H. Booske, I. Dobson.
  Comparison of Four Multifrequency Traveling Wave Tube Models.
  42$^{\rm nd}$ Annual Meeting of the American Physical Society Division
  of Plasma Physics combined with the 10$^{\rm th}$ International Congress
  on Plasma Physics, October 2000.
\item J.G. W\"ohlbier, I. Dobson, J.H. Booske, J.E. Scharer.
  Simplified Model of a Multitoned Traveling Wave Tube Amplifier.
  41$^{\rm st}$ Annual Meeting of the American Physical Society Division
  of Plasma Physics, November 1999.
\item J.G. W\"ohlbier, I. Dobson, J.H. Booske, J.E. Scharer.
  Simplified Model of a Multitoned Traveling Wave Tube Amplifier.
  26$^{\rm th}$ IEEE Conference on Plasma Science, June 1999.
\item J.G. W\"ohlbier, J.H. Booske, J.E. Scharer, I. Dobson, B. VanVeen.
  Characterization of Nonlinearities in Multitone Helix Traveling Wave Tube
  Amplifiers.
  40$^{\rm th}$ Annual Meeting of the American Physical Society Division
  of Plasma Physics, November 1998.
\end{itemize}

%
%%{\bf\large Theses}
\begin{enumerate}
\item J.G. W\"ohlbier. Nonlinear Distortion and Suppression in
  Traveling Wave Tubes: Insights and Methods. PhD Thesis, University
  of Wisconsin -- Madison, 2003.
\item J.G. W\"ohlbier. Modeling and Analysis of a Traveling Wave Tube
  Under Multitone Excitation. Master's Thesis, University of
  Wisconsin -- Madison, 2000.
\end{enumerate}

%
%{\bf\large Professional Societies}
%\begin{itemize}
%\item Institute of Electrical and Electronics Engineers -- Member
%\item American Physical Society -- Member
%\end{itemize}
\end{document}
