\documentclass[12pt]{article}
\setlength{\oddsidemargin}{-0.2in}
\setlength{\textwidth}{6.5in}
\setlength{\topmargin}{-0.5in}
\setlength{\textheight}{9.5in}

\def\rage{{\sf RAGE}}

\usepackage{fancyhdr}
\renewcommand{\headrulewidth}{0.0mm}
\renewcommand{\footrulewidth}{0.0mm}
\fancyhf{}
\fancyhead[R]{April 2011}
\pagestyle{fancy}
\renewcommand{\labelitemii}{$\Rightarrow$}

\begin{document}
\begin{center}
  \textbf{John G. Wohlbier}\\
    Technical Staff Member\\
    Los Alamos National Laboratory\\
    Computational Physics Group, CCS-2\\
  PO Box 1663, MS D413\\
  Los Alamos, NM 87545\\
  (505)664-0794\\
%  {\tt wohlbier@lanl.gov}\\
  {\tt johnwohlbier@gmail.com}
\end{center}

\noindent\textbf{Education}
\begin{itemize}
\item Ph.D. in Electrical and Computer Engineering, University
  of Wisconsin--Madison, May 2003.
  Thesis: Nonlinear Distortion and Suppression in Traveling Wave
  Tubes: Insights and Methods. Advisors: John Booske and Ian Dobson.
\item Master of Science in Electrical Engineering,
  University of Wisconsin--Madison, May 2000.
  Thesis: Modeling and Analysis of a Traveling Wave Tube Under Multitone
  Excitation.
\item Bachelor of Science in Electrical Engineering,
  University of Wisconsin--Madison, August 1993.
\end{itemize}

\noindent\textbf{Experience}
\begin{itemize}
\item {\bf 2005--present:} Technical Staff Member, Computational Physics
  Group (CCS-2), Los Alamos National Laboratory.
  \begin{itemize}
  \item \textbf{Project Lead: Multi-physics on Multi-core}
    Lead project focused on developing multi-physics
    application codes for modern supercomputer architectures,
    including the Roadrunner machine (first to break 1PF/s barrier)
    and GPGPU accelerated architectures.
  \item \textbf{Multi-physics developer} Develop and implement
    multi-physics models, solvers, and algorithms including, e.g.,
    \emph{shock waves, plasma physics, radiation transport, nuclear
      reaction networks, self-gravitation} 
  \end{itemize}
\item {\bf 2003--2005:} Agnew National Security Postdoctoral Fellow,
  Los Alamos National Laboratory.
  \begin{itemize}
  \item Applied novel analytic and numerical methods to study wave
  breaking, saturation,
  efficiency, and linearity in Traveling Wave Tubes and related
  devices. Work was on forefront of methods in nonlinear
  waves and shock physics.
  Large scale computations developed and
  performed on Los Alamos parallel computing resources.
  \end{itemize}
\item {\bf 2003:} Postdoctoral Scholar, Nuclear Engineering and Engineering
  Physics, University of Wisconsin--Madison.
  \begin{itemize}
  \item Performed theoretical and numerical research in stabilization
  of ballooning instabilities in 3-d plasmas for magnetic fusion
  applications.
  \end{itemize}
\item {\bf 1997--2003:} Research Assistant, Electrical and Computer
  Engineering, University of Wisconsin--Madison.
  \begin{itemize}
  \item Performed research in nonlinear behavior of Traveling Wave Tube
  amplifiers. Research included computer simulations and theoretical
  analyses of Traveling Wave Tubes and related devices and resulted in
  eight refereed journal articles.
  \end{itemize}
\item {\bf 1994--1997:} Engineer, Soft Switching Technologies,
  Middleton, Wisconsin.
  \begin{itemize}
  \item Electrical and mechanical engineering design for start-up company.
  Involved in all facets of design and production of power quality
  products.
  \end{itemize}
\item {\bf 1993--1994:} Research Assistant, Electrical and Computer
  Engineering, University of Wisconsin--Madison.
  \begin{itemize}
  \item Performed and published research in semiconductor lasers.
  \end{itemize}
\end{itemize}

\fancyhead[R]{John G.~Wohlbier, April 2011}

\noindent\textbf{Mathematics Skills}
\begin{itemize}
\item Partial Differential Equations (PDEs)
  \begin{itemize}
  \item Shock capturing schemes for hyperbolic PDEs
  \item Familiarity with multi-grid schemes for elliptic and parabolic
    PDEs
  \end{itemize}
\item Ordinary Differential Equations (ODEs)
  \begin{itemize}
  \item Implicit and explict methods for stiff ODEs, e.g., nuclear
    reaction networks, exact Riemann solvers, etc.
  \end{itemize}
\item Multi-physics coupling of PDEs and ODEs: operator splitting in
  time; fully implicit, fully coupled non-linear solves; hybrid
  implicit-explicit schemes for high order time accuracy
\item Training in measure-theoretic probability
\end{itemize}

\noindent\textbf{Computing Skills}
\begin{itemize}
\item Code development for large scale science applications on
  supercomputers
  \begin{itemize}
  \item Traditional massively parallel architectures: homogeneous
    multi-core nodes connected via infiniband
  \item Modern massively multi-threaded architectures: hetergeneous
    multi-core (CellBE processor in the Roadrunner supercomputer)
    and homogeneous many-core (Nvidia GPGPU)
  \end{itemize}
\item Languages: C++, C, Python, Fortran, CUDA, OpenCL
\item Libraries: Message Passing Interface (MPI), DaCS and libspe2
  (IBM libraries for the CellBE and Roadrunner), various other
  scientific computing libraries
\item Operating Systems: Linux, Macintosh, Windows
\end{itemize}

\noindent{\bf Selected Refereed Journal Publications}
\begin{itemize}
\item T.O.~Masser, \textbf{J.G.~Wohlbier}, and R.B.~Lowrie. Shock wave structure
  for a fully ionized plasma. {\it Shock Waves}, accepted.
\item R.G.~McClarren and \textbf{J.G.~Wohlbier}. Solutions for
  ion-electron-radiation coupling with radiation and electron
  diffusion. {\it J.~Quant.~Spectrosc.~Radiat.~Transfer}, 112:119--130
  (2011).
\item \textbf{J.G.~Wohlbier}. Phase distortion mechanisms in linear beam
  vacuum devices. {\it IEEE Trans.~Plasma Sci.}, Vol.~33, no.~3,
  2005.
\item \textbf{J.G.~Wohlbier}, S.~Jin, S.~Sengele. Eulerian calculations of
  wave breaking and multi-valued solutions in a traveling wave
  tube. {\it Physics of Plasmas} {\bf 12}, 023106 (2005).
\item J.G. Wohlbier and J.H. Booske. Nonlinear space charge wave theory
  of distortion in a klystron. {\it IEEE Trans.~Electron Devices,}
  Vol.~52, no.~5, 2005.
\item A. Singh, J.E. Scharer, J.H. Booske, and \textbf{J.G.~Wohlbier}.
  Second and third-order signal injection for nonlinear
  distortion suppression in a traveling wave tube. {\it IEEE
    Trans.~Electron Devices,} Vol.~52, no.~5, 2005.
\item Section 3, ``Theoretical Principles,'' in Chapter 9,
  ``How to Achieve Linear Amplification,'' in Barker, R.J., Booske,
  J.H., Luhmannn, N.C., Nusinovich, G.S.
  {\it Modern Microwave and Millimeter Wave Power Electronics:} IEEE
  Press. ISBN: 978-0-471-68372-8, 2005.
\item A. Singh, \textbf{J.G.~Wohlbier}, J.H. Booske, and
  J.E. Scharer. Experimental Verification of the Mechanisms for
  Nonlinear Harmonic Growth and Suppression by Harmonic Injection in a
  Traveling Wave Tube. {\it Phys.~Rev.~Lett.} 92, 205005 (2004).
\item X. Li, \textbf{J.G.~Wohlbier}, S. Jin, and J.H. Booske.
  Eulerian Method for Computing Multi-valued solutions of the Euler-Poisson
  Equations and Application to Wave Breaking in Klystrons.
  {\it Phys.~Rev.~E} 70, 016502 (2004).
\item \textbf{J.G.~Wohlbier} and J.H. Booske.
  Mechanisms of Phase Distortion in a Traveling Wave Tube.
  {\it Phys.~Rev.~E} 69, 066502 (2004).
\item \textbf{J.G.~Wohlbier}, J.H. Booske, and I. Dobson.
  On the Physics of Harmonic Injection in a Traveling Wave Tube.
  {\it IEEE Trans.~Plasma Sci.}, Vol.~32, No.~3, (2004).
\item \textbf{J.G.~Wohlbier}, I. Dobson, and J.H. Booske.
  Generation and growth rates of nonlinear distortions in a traveling
  wave tube. {\it Phys.~Rev.~E} 66, 56504 (2002).
\item \textbf{J.G.~Wohlbier}, J.H. Booske, and I. Dobson.
  The Multifrequency Spectral Eulerian (MUSE) Model of a Traveling Wave
  Tube. {\it IEEE Trans.~Plasma Sci.} Vol.~30, no.~3, June 2002.
\item T. Zhang, \textbf{J.G.~Wohlbier}, K. D. Choquette, N. Tabatabaie.
  Microcavity Vacuum-Field Configuration and the Spontaneous Emission
  Power. {\it IEEE Journal on Selected Topics in Quantum Electronics.}
  Vol.~1, no.~2, pp.~601--605, 1995.
\end{itemize}


\noindent\textbf{Honors and Awards}
\begin{itemize}
\item 2003 Agnew National Security Postdoctoral Fellow, Los Alamos
  National Laboratory. Highly competitive fellowship, only three
  awarded out of nearly 400 post-docs.
\item 1993 Teaching Assistant of the Year Award. University of
  Wisconsin--Madison.
  An award decided by student vote in the department of Electrical and
  Computer Engineering.
\end{itemize}

\noindent\textbf{Patent}
\begin{itemize}
\item U.S. Patent 6,087,916. Cooling of coaxial winding transformers in
  high power applications. N.H. Kutkut, D.M. Divan, \textbf{J.G.~Wohlbier},
  R.W. Gascoigne.
\end{itemize}

\end{document}
